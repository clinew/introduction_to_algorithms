\documentclass{article}

\usepackage{amsmath}

\begin{document}

\section*{Problem 2-3}

\noindent\begin{enumerate}
	\item[\textbf{\textit{a.}}]
		The running time of the code fragment is $\Theta(n)$.

	\item[\textbf{\textit{b.}}]
		See \texttt{p2-2.c} for an implementation in C. The running time of the algorithm is $\Theta(n^2)$. This is \textit{much} worse than Horner's Rule.

	\item[\textbf{\textit{c.}}]
		\textit{Initialization}: At loop initialization, $i=n$, giving:
		\begin{eqnarray*}
			\sum_{k=0}^{n-i+1} a_{k+i+1} x^k & = & \sum_{k=0}^{-1} a_{k+n+1} x^k \\
			& = & 0
		\end{eqnarray*}
		\textit{Maintenance}: Assume that at $i$:
		\begin{eqnarray*}
			y & = & \sum_{k=0}^{n-(i+1)} a_{k+i+1} x^k
		\end{eqnarray*}
		Then prove that at $i-1$:
		\begin{eqnarray*}
			y & = & \sum_{k=0}^{n-i} a_{k+i} x^k
		\end{eqnarray*}
		From the structure of the loop in the pseudocode, we know that at $i-1$:
		\begin{eqnarray*}
			y & = & a_i + x \sum_{k=0}^{n-(i+1)} a_{k+i+1} x^k \\
			& = & a_i + x \left ( a_{i+1} + a_{i+2} + \cdots + a_{n+1} x^{n-i-2} + a_n x^{n-i-1} \right ) \\
			& = & a_i + a_{i+1} x + a_{i+2} x^2 + \cdots + a_{n-1} x^{n-i-1} + a_n x^{n-1} \\
			& = & \sum_{k=0}^{n-1} a_{k+i} x^k
		\end{eqnarray*}
		\textit{Termination}: At loop termination, $i=-1$, giving:
		\begin{eqnarray*}
			y & = & \sum_{k=0}^{n-(i+1)} a_{k+i+i} x^k \\
			& = & \sum_{k=0}^n a_k x^k
		\end{eqnarray*}

	\item[\textbf{\textit{d.}}] By the Loop Invariant's Termination Condition and Horner's Rule, we know that the given code fragment correctly evaluates the polynomial.
\end{enumerate}

\end{document}

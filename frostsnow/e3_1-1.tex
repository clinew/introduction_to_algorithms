\documentclass{article}

\usepackage{amsmath}

\begin{document}

\section*{Exercise 3.1-1}

By definition of $\Theta$:
\begin{eqnarray*}
	\Theta(f(n) + g(n)) = \{h(n) \text{: there exist positive constants } c1, c2, \text{ and } n_0 \text{ such that } \\
	0 \leq c_1 h(n) \leq f(n) + g(n) \leq c_2 h(n) \text{ for all } n \geq n_0\}.
\end{eqnarray*}
We know by the definition of max that:
\begin{eqnarray*}
	\text{max}(f(n), g(n)) \leq f(n) + g(n) \leq c_2 h(n)
\end{eqnarray*}
So choose $c_1$ such that:
\begin{eqnarray*}
	c_1 h(n) \leq \text{max}(f(n),g(n)) \text{ for all } n_0 \text{ sufficiently large}
\end{eqnarray*}
I'm not sure if this is a mathematically-valid proof; what I'm trying to prove makes sense, but actually proving it is more difficult.

\end{document}

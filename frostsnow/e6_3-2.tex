\documentclass{article}

\usepackage{amsmath}

\begin{document}

\section*{Exercise 6.3-2}

On any given call to \texttt{MAX-HEAPIFY}, a small element may ``percolate'' down the tree many times, but a large element may only percolate up once, and will not percolate beyond the height $h$ that the call was made at. If when the initial call is made the largest element of the heap is at a height $h > 2$, that element will at most move to height $h = 1$, but will not be able to move to its proper location at $h = 0$, as all subsequent calls will be made at height $h \geq 1$.

\end{document}

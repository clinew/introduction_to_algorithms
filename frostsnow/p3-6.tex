\documentclass{article}

\usepackage{amsmath}
\usepackage[usenames,dvipsnames]{color}

\begin{document}

\section*{Problem 3-6}

\noindent\begin{center}\begin{tabular}{c|c c|c}
	& $f(n)$ & $c$ & $f_c^*(n)$ \\ \hline
	\textbf{\textit{a.}} & $n - 1$ & $0$ & $n$ \\
	\textbf{\textit{b.}} & $\log_2(n)$ & $1$ & $\log_2^*(n)$ \\
	\textbf{\textit{c.}} & $\frac{n}{2}$ & $1$ & $\log_2(n)$ \\
	\textbf{\textit{d.}} & $\frac{n}{2}$ & $2$ & $\log_2(n) - 1$ \\
	\textbf{\textit{e.}} & $\sqrt{n}$ & $2$ & $\log_2(\log_2(n))$ \\
	\textbf{\textit{f.}} & $\sqrt{n}$ & $1$ & $\infty$ \\
	\textbf{\textit{g.}} & $n^{\frac{1}{3}}$ & $2$ & $\log_3(\log_2(n))$ \\
	% Fuck this problem. Three months and no clue how to solve it...
	% Still no idea.  https://walkccc.me/CLRS/Chap03/Problems/3-6/ lists a
	% \omega(lg lg n) and o(lg n).  Is that "as tight as possible"?  The
	% bounds seem accurate, but I don't see any justification for the
	% answer.
	\textbf{\textit{h.}} & $\frac{n}{\log_2(n)}$ & $2$ & ??? \\
\end{tabular}
\end{center}

\end{document}

\documentclass{article}

\usepackage{amsmath}

\begin{document}

\section*{Problem 3-4}

\noindent\textbf{\textit{a.}} By definition of $O$-notation, $f(n) = O(g(n))$ if there exist constants $c_0$ and $n_0$ such that $0 \leq f(n) \leq c_0 g(n)$ for all $n \geq n_0$. Then, assuming for the sake of contradiction that if $g(n) = O(f(n))$, there exist positive constants $c_1$ and $n_1$ such that $0 \leq g(n) \leq c_1 f(n)$ for all $n \geq n_1$. But suppose that $f(n) = o(g(n))$, then for any constant $c_2 > 0$, there exists an $n_2$ such that $0 \leq f(n) < c_2 g(n)$ for all $n \geq n_2$. However, since we \textit{assumed} that:
\begin{eqnarray*}
	g(n) & \leq & c_1 f(n) \\
	\frac{g(n)}{c_1} & \leq & f(n)
\end{eqnarray*}
and $\frac{1}{c_1}$ is a positive constant $c_2 > 0$ and we \textit{know} that $f(n) < c_2 g(n)$, then:
\begin{eqnarray*}
	f(n) < c_2 g(n) \leq f(n)
\end{eqnarray*}
which makes no sense and thus contradicts the assumption.

Therefore, $f(n) = O(g(n))$ does not imply $g(n) = O(g(n))$.

\noindent\textbf{\textit{b.}} By definition of $\Theta$-notation, $f(n) + g(n) = \Theta(\min(f(n),g(n)))$ if there exist positive constants $c_1$, $c_2$, and $n_0$ such that $0 \leq c_1 \min(f(n),g(n)) \leq f(n) + g(n) \leq c_2 \min(f(n),g(n))$. Assume for the sake of contradiction that this is true. Then suppose that $f(n) = o(g(n))$; by definition of $o$-notation, for all $c_3 > 0$ there exists some $n_3 > 0$ such that $0 \leq f(n) < c_3 g(n)$ for all $n \geq n_3$. Then:
\begin{eqnarray*}
	0 \leq c_1 \min(f(n),g(n)) \leq f(n) + g(n) & \leq & c_2 \min(f(n),g(n)) \\
	f(n) + g(n) & \leq & c_2 \min(f(n),g(n)) \leq c_2 (f(n) + g(n)) \\
	1 & \leq & c_2 \\
	c_2 & \geq & 1
\end{eqnarray*}
Furthermore, for all $c_3 \geq 1$, if $n \geq n_3$ then by the assumption:
\begin{eqnarray*}
	f(n) + g(n) & \leq & c_2 \min(f(n),g(n)) \leq c_2 \min(f(n),c_3 g(n)) \\
	f(n) + g(n) & \leq & c_2 \min(f(n),c_3 g(n)) \\
	f(n) + g(n) & \leq & c_2 f(n) \text{ (By $f(n) = o(g(n))$.)} \\
	g(n) & \leq & c_2 f(n) - f(n) \\
	g(n) & \leq & f(n)(c_2 - 1) \\
	\frac{g(n)}{c_2 - 1} & \leq & f(n)
\end{eqnarray*}
...but for all $c_2 \geq 1, \frac{1}{c_2 - 1}$ is a constant greater than zero which contradicts the fact that $f(n) = o(g(n))$.

Therefore, $f(n) + g(n) \neq \Theta(\min(f(n),g(n)))$.

\noindent\textbf{\textit{c.}} By definition of $O$-notation,  $f(n) = O(g(n))$ if there exist positive constants $c$ and $n_0$ such that $0 \leq f(n) \leq c g(n)$ for all $n \geq n_0$. Then:
\begin{eqnarray*}
	0 \leq \log_2(f(n)) & \leq & \log_2(c g(n)) \\
	0 \leq \log_2(f(n)) & \leq & \log_2(c) + \log_2(g(n))
\end{eqnarray*}
As $n \rightarrow \infty$, $\log_2 c$ becomes insignificant, leaving:
\begin{eqnarray*}
	0 \leq \log_2(f(n)) \leq \log_2(g(n))
\end{eqnarray*}
Thus, $f(n) = O(g(n))$ implies $\log_2(f(n)) = O(\log_2(g(n)))$.

\noindent\textbf{\textit{d.}} Assume for the sake of contradiction that $f(n) = O(g(n))$ implies $ 2^{f(n)} = O(2^{g(n)})$. Then, by definition of $O$, there exists a positive constant $c_0$ such that:
\begin{eqnarray*}
	2^{f(n)} & \leq & c_0 2^{g(n)} \\
	\frac{2^{f(n)}}{2^{g(n)}} & \leq & c_0 \\
	2^{f(n) - g(n)} & \leq & c_0
\end{eqnarray*}
Now suppose that $f(n) = 3n$ and $g(n) = 2n$. Then for all $c > \frac{3}{2}$ and $n \geq 0$, $f(n) = O(g(n))$. However, solving for $c_0$ in the preceding equation array gives:
\begin{eqnarray*}
	2^{f(n) - g(n)} & \leq & c_0 \\
	2^{3n - 2n} & \leq & c_0 \\
	2^n & \leq & c_0
\end{eqnarray*}
Which is a contradiction, since a constant always has a value smaller than that of a variable which goes towards infinity (let alone a variable which goes towards infinity expoentially).

Therefore, $f(n) = O(g(n))$ does not imply $2^{f(n)} = O(2^{g(n)})$.

\noindent\textbf{\textit{e.}} By definition of $O$-notation, $f(n) = O((f(n))^2)$ if there exist positive constants $c$ and $n_0$ such that $0 \leq f(n) \leq c (f(n))^2$ for all $n \geq n_0$. Suppose for the sake of contradiction that $f(n) = O((f(n))^2)$ and that the $n^{\text{th}}$ term of $f(n)$ represents the $n^{\text{th}}$ term of the Haromic Series, i.e. $f(n) = \frac{1}{n}$. Then:
\begin{eqnarray*}
	0 \leq f(n) & \leq & c (f(n))^2 \\
	0 \leq 1 & \leq & c f(n) \\
	0 \leq \frac{1}{f(n)} & \leq & c \\
	0 \leq n & \leq & c
\end{eqnarray*}
But $n$ cannot possibly be less than some constant $c$ for some arbitrarily large $n$, thus the assumption is contradicted.

Therefore, $f(n) \neq O((f(n))^2)$ for some arbitrary, asymptotically positive function $f(n)$.

\noindent\textbf{\textit{f.}} By definition of $O$, $f(n) = O(g(n))$ if there exist positive constants $c_0$ and $n_0$ such that $0 \leq f(n) \leq c_0 g(n)$ for all $n \geq n_0$. By definition of $\Omega$, $g(n) = \Omega(f(n))$ if there exist positive constants $c_1$ and $n_1$ such that $0 \leq g(n) \leq c_1 f(n)$ for all $n \geq n_1$. Assuming that $f(n) = O(g(n))$:
\begin{eqnarray*}
	0 \leq f(n) & \leq & c_0 g(n) \\
	0 \leq \frac{f(n)}{g(n)} & \leq & c_0 \\
	0 \leq \frac{1}{g(n)} & \leq & \frac{c_0}{f(n)} \\
	0 \leq \left ( \frac{1}{g(n)} \right )^{-1} & \leq & \left ( \frac{c_0}{f(n)} \right )^{-1} \\
	0 \leq g(n) & \leq & \frac{f(n)}{c_0}
\end{eqnarray*}
Which, letting $c_1 = \frac{1}{c_0}$, fits the definition of $g(n) = \Omega(f(n))$.

Therefore $f(n) = O(g(n))$ implies $g(n) = \Omega(f(n))$.

\noindent\textbf{\textit{g.}} By definition of $\Theta$, $f(n) = \Theta(f(n/2))$ if there exist positive constants $c_1$, $c_2$, and $n_0$ such that $0 \leq c_1 f(n/2) \leq f(n) \leq c_2 f(n/2)$. Suppose for the sake of counter-example that $f(n) = 2^n$, then:
\begin{eqnarray*}
	0 \leq c_1 f(n/2) \leq f(n) \leq c_2 f(n/2) \\
	0 \leq c_1 \leq \frac{f(n)}{f(n/2)} \leq c_2 \\
	0 \leq c_1 \leq \frac{2^n}{2^{n/2}} \leq c_2 \\
	0 \leq c_1 \leq 2^{n - n/2} \leq c_2 \\
	0 \leq c_1 \leq 2^{n/2} \leq c_2
\end{eqnarray*}
However, $2^{n/2}$ cannot be less than $c_2$ for some arbitrarily large value of $n$.

Therefore $f(n) \neq \Theta(f(n/2))$ for some arbitrary asymptotically positive $f(n)$.

\noindent\textbf{\textit{h.}} By definition of $\Theta$, $f(n) + o(f(n)) = \Theta(f(n))$ if there exist positive constants $c_0$, $c_1$, and $n_0$ such that:
\begin{eqnarray*}
	0 \leq c_0 f(n) \leq f(n) + o(f(n)) \leq c_1 f(n)
\end{eqnarray*}
Suppose for the sake of counterexample that $f(n) = n$ and $o(f(n)) = n^2$. Then:
\begin{eqnarray*}
	0 \leq c_0 f(n) \leq f(n) + o(f(n)) \leq c_1 f(n) \\
	0 \leq c_0 n \leq n + n^2 \leq c_1 n \\
	0 \leq c_0 \leq 1 + n \leq c_1
\end{eqnarray*}
However, $n + 1$ cannot be less than or equal to some positive constant for arbitrarily large value of $n$.

Therefore, $f(n) + o(f(n)) \neq \Theta(f(n))$.

\end{document}

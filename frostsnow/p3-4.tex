\documentclass{article}

\usepackage{amsmath}

\begin{document}

\section*{Problem 3-4}

\noindent\textbf{\textit{a.}} By definition of $O$-notation, $f(n) = O(g(n))$ if there exist constants $c_0$ and $n_0$ such that $0 \leq f(n) \leq c_0 g(n)$ for all $n \geq n_0$. Then, assuming for the sake of contradiction that if $g(n) = O(f(n))$, there exist positive constants $c_1$ and $n_1$ such that $0 \leq g(n) \leq c_1 f(n)$ for all $n \geq n_1$. But suppose that $f(n) = o(g(n))$, then for any constant $c_2 > 0$, there exists an $n_2$ such that $0 \leq f(n) < c_2 g(n)$ for all $n \geq n_2$. However, since we \textit{assumed} that:
\begin{eqnarray*}
	g(n) & \leq & c_1 f(n) \\
	\frac{g(n)}{c_1} & \leq & f(n)
\end{eqnarray*}
and $\frac{1}{c_1}$ is a positive constant $c_2 > 0$ and we \textit{know} that $f(n) < c_2 g(n)$, then:
\begin{eqnarray*}
	f(n) < c_2 g(n) \leq f(n)
\end{eqnarray*}
which makes no sense and thus contradicts the assumption.

Therefore, $f(n) = O(g(n))$ does not imply $g(n) = O(g(n))$.

\noindent\textbf{\textit{b.}}

\noindent\textbf{\textit{c.}}

\noindent\textbf{\textit{d.}}

\noindent\textbf{\textit{e.}}

\noindent\textbf{\textit{f.}}

\noindent\textbf{\textit{g.}}

\noindent\textbf{\textit{h.}}

\end{document}

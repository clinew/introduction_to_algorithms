\documentclass{article}

\usepackage{amsmath}
\usepackage[usenames,dvipsnames]{color}

\begin{document}

\section*{Problem 3-5}

\noindent\textbf{\textit{a.}} The best solution I've managed to find for this problem is a proof by parts. The question makes the most sense with regards to ``multi-way'' functions, for example:
\begin{eqnarray*}
	f(n) =
	\begin{cases}
		x^2 & \text{if } x \% 2 = 0,\\
		x & \text{if } x \%2 = 1.
	\end{cases}
\end{eqnarray*}
In this case, $f(n)$ is both $O(n^2)$ and $\Omega(n)$.

In order to make this proof, I'm going to make a fairly large, but, in my opinion, reasonable, assumption. That assumption is that, for any function $f(n)$, there exist functions $f_a(n)$ and $f_b(n)$ such that $f(n) = O(f_a(n))$ and $f(n) = \Omega(f_b(n))$.

Now, suppose that there exist two asymptotically non-negative functions $f(n)$ and $g(n)$. Then there also exist functions $f_a(n)$, $f_b(n)$, $g_a(n)$, and $g_b(n)$ such that:
\begin{eqnarray*}
	f(n) & = & O(f_a(n)) \\
	f(n) & = & \Omega(f_b(n)) \\
	g(n) & = & O(g_a(n)) \\
	g(n) & = & \Omega(g_b(n))
\end{eqnarray*}
So, construct a truth table of all possible relations between the aforementioned functions and their respective notations:


\noindent\begin{tabular}{c c c c|c|c|c}
	$f_a = ?g_a$ & $f_b = ?g_a$ & $f_a = ?g_b$ & $f_b = ?g_b$ & $f(n) = O(g(n))$ & $f(n) = \overset{\small\infty}{\Omega}(g(n))$ & $f(n) = \Omega(g(n))$ \\ \hline
	{\color{Gray}$\omega$} & $\omega$ & {\color{Gray}$\omega$} & {\color{Gray}$\omega$} & {\color{Red}No} & {\color{Green}Yes} & {\color{Green}Yes} \\
	$\omega$ & $\Theta$ & {\color{Gray}$\omega$} & $\omega$ & {\color{Red}No} & {\color{Green}Yes} & {\color{Green}Yes} \\
	$\Theta$ & $\Theta$ & {\color{Gray}$\omega$} & $\omega$ & {\color{Red}No} & {\color{Green}Yes} & {\color{Green}Yes} \\
	$\omega$ & $o$ & {\color{Gray}$\omega$} & $\omega$ & {\color{Red}No} & {\color{Green}Yes} & {\color{Red}No} \\
	$\Theta$ & $o$ & {\color{Gray}$\omega$} & $\omega$ & {\color{Red}No} & {\color{Green}Yes} & {\color{Red}No} \\
	$o$ & {\color{Gray}$o$} & {\color{Gray}$\omega$} & $\omega$ & {\color{Red}No} & {\color{Green}Yes} & {\color{Red}No} \\
	$\omega$ & $\Theta$ & $\omega$ & $\Theta$ & {\color{Red}No} & {\color{Green}Yes} & {\color{Green}Yes} \\
	$\omega$ & $o$ & $\omega$ & $\Theta$ & {\color{Red}No} & {\color{Green}Yes} & {\color{Red}No} \\
	$\Theta$ & $o$ & $\omega$ & $\Theta$ & {\color{Red}No} & {\color{Green}Yes} & {\color{Red}No} \\
	$o$ & {\color{Gray}$o$} & $\omega$ & $\Theta$ & {\color{Red}No} & {\color{Green}Yes} & {\color{Red}No} \\
	$\omega$ & {\color{Gray}$o$} & $\omega$ & $o$ & {\color{Red}No} & {\color{Green}Yes} & {\color{Red}No} \\
	$\Theta$ & {\color{Gray}$o$} & $\omega$ & $o$ & {\color{Red}No} & {\color{Green}Yes} & {\color{Red}No} \\
	$o$ & {\color{Gray}$o$} & $\omega$ & $o$ & {\color{Red}No} & {\color{Green}Yes} & {\color{Red}No} \\
	$\Theta$ & $\Theta$ & $\Theta$ & $\Theta$ & {\color{Green}Yes} & {\color{Green}Yes} & {\color{Green}Yes} \\
	$o$ & {\color{Gray}$o$} & $\Theta$ & $\Theta$ & {\color{Green}Yes} & {\color{Green}Yes} & {\color{Red}No} \\
	$\Theta$ & {\color{Gray}$o$} & $\Theta$ & $o$ & {\color{Green}Yes} & {\color{Green}Yes} & {\color{Red}No} \\
	$o$ & {\color{Gray}$o$} & $\Theta$ & $o$ & {\color{Green}Yes} & {\color{Green}Yes} & {\color{Red}No} \\
	{\color{Gray}$o$} & {\color{Gray}$o$} & $o$ & {\color{Gray}$o$} & {\color{Green}Yes} & {\color{Red}No} & {\color{Red}No} \\
\end{tabular}
From inspecting the truth table, it is apparent that either $f(n) = O(g(n))$, $f(n) = \overset{\small\infty}{\Omega}(g(n))$, or both, but this is not true if we use $\Omega$ in place of $\overset{\small\infty}{\Omega}$.

\noindent\textbf{\textit{b.}} The set of all functions such that $f(n) = \overset{\small\infty}{\Omega}(g(n))$ is a superset of all functions such that $f(n) = \Omega(g(n))$. The former notation may be useful if one needs a less strict approximation of a lower bound.

\noindent\textbf{\textit{c.}} Assume that $f(n) = \Theta(g(n))$. Then, by definition of $\Theta$, there exist positive constants $c_1$, $c_2$, and $n_0$ such that $0 \leq c_1 g(n) \leq f(n) \leq c_2 g(n)$ for all $n \geq n_0$. Assume for the sake of contradiction that $f(n) = O'(g(n))$, then, by defintiion of $O'$, there exist some positive constants $c_3$ and $n_1$ such that $0 \leq | f(n) | \leq c_3 g(n)$ for all $n \geq n_1$. Suppose then that $f(n) = -n^3$, $g(n) = -n^3$, $c_1 \geq 1$, and $c_2 \leq 1$, then $f(n) = \Theta(g(n))$, but $0 \leq | f(n) | \leq c_3 g(n)$ is false for all values of $n$ besides $0$ and any value of $c_3$, thus $f(n) = \Theta(g(n))$ does not imply $f(n) = O'(g(n))$.

However, using the previous definition of $f(n) = \Theta(g(n))$, now assume that $f(n) = O'(g(n))$ and $f(n) = \Omega(g(n))$. Then there exist some positive constants $c_3$ and $n_1$ such that $0 \leq | f(n) | \leq c_3 g(n)$ for all $n \geq n_1$ and there exist some positive constants $c_4$ and $n_2$ such that $0 \leq c_4 g(n) \leq f(n)$ for all $n \geq n_2$. Since $f(n) \leq | f(n) |$, it follows that $0 \leq c_4 g(n) \leq f(n) \leq | f(n) | \leq c_3 g(n)$ for all $n \geq \max(n_1, n_2)$; letting $c_1 = c_4$, $c_2 = c_3$, and $n_0 = \max(n_1, n_2)$ means that $f(n) = \Theta(g(n))$, thus $f(n) = O'(g(n))$ and $f(n) = \Omega(g(n))$ implies that $f(n) = \Theta(g(n))$.

Therefore, Theorum 3.1 becomes, ``for any two functions, $f(n)$ and $g(n)$, if $f(n) = O'(g(n))$ and $f(n) = \Omega(g(n))$ then $f(n) = \Theta(g(n))$.'' The two expression are not, however, equivalent.

\noindent\textbf{\textit{d.}} Begin by writing out the definitions. First, $f(n) = \overset{\small\sim}{\Omega}(g(n))$ if there exist positive constants $c_0$, $k_0$, and $n_0$ such that $0 \leq c_0 g(n) \log_2^{k_0}(n) \leq f(n)$ for all $n \geq n_0$. Second, $f(n) = \overset{\small\sim}{O}(g(n))$ if there exist positive constants $c_1$, $k_1$, and $n_1$ such that $0 \leq f(n) \leq c_1 g(n) \log_2^{k_1}(n)$ for all $n \geq n_1$. Last, $f(n) = \overset{\small\sim}{\Theta}(g(n))$ if there exist positive constants $c_2$, $k_2$, $c_3$, $k_3$, and $n_2$ such that $0 \leq c_2 g(n) \log_2^{k_2}(n) \leq f(n) \leq c_3 g(n) \log_2^{k_3}(n)$ for all $n \geq n_2$. Next, begin the Proof of Equivalence by assuming that $f(n) = \overset{\small\sim}{\Theta}(g(n))$. Then let $c_0 = c_2$, $k_0 = k_2$, and $n_0 = n_2$, at which point $f(n) = \overset{\small\sim}{\Omega}(g(n))$. Likewise, let $c_1 = c_3$, $k_1 = k_3$, and $n_1 = n_3$, at which point $f(n) = \overset{\small\sim}{O}(g(n))$. Thus $f(n) = \overset{\small\sim}{\Theta}(g(n))$ implies $f(n) = \overset{\small\sim}{O}(g(n))$ and $f(n) = \overset{\small\sim}{\Omega}(g(n))$. Finally, finish the Proof of Equivalence by assuming that $f(n) = \overset{\small\sim}{O}(g(n))$ and $f(n) = \overset{\small\sim}{\Omega}(g(n))$. Then $0 \leq c_0 g(n) \log_2^{k_0}(n) \leq f(n) \leq c_1 g(n) \log_2^{k_1}(n)$ for all $n \geq \max(n_0, n_1)$. Letting $c_2 = c_0$, $k_2 = k_0$, $c_3 = c_1$, $k_3 = k_1$, and $n_2 = \max(n_0, n_1)$ thus shows that $f(n) = \overset{\small\sim}{O}(g(n))$ and $f(n) = \overset{\small\sim}{\Omega}(g(n))$ implies that $f(n) = \overset{\small\sim}{\Theta}(g(n))$. Therefore, $f(n) = \overset{\small\sim}{\Theta}(g(n))$ if and only if $f(n) = \overset{\small\sim}{O}(g(n))$ and $f(n) = \overset{\small\sim}{\Omega}(g(n))$.
\end{document}

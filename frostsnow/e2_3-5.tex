\documentclass{article}

\usepackage{amsmath}

\begin{document}

\section*{Exercise 2.3-5}

An example of a binary search may be found in \texttt{p1-1.c}.
\\ \\
\noindent~A binary search terminates when the remaining sequence is 1 (or $< 1$, if $n$ is not a power of 2). Suppose that it takes $i$ or more iterations to solve the binary search, then, since the sequency length is divided by half for each iteration $i$, the number of iterations it would take to solve a sequence of length $n$ would be:
\begin{equation*}
	\frac{n}{2^i} < 1
\end{equation*}
Solving for the number of iterations needed to solve a sequence of length $n$ gives:
\begin{eqnarray*}
	n & < & 2^i \\
	\log_2(n) & < & \log_2(2^i) \\
	\log_2(n) & < & i
\end{eqnarray*}
Thus the running time of a binary search is $\Theta(\log_2(n))$.

\end{document}

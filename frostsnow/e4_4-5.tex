Assume a base case of $T(1) = 1$.  The recurrence tree then looks something like:

\begin{center}
\begin{forest}
[$n$,
	[$n-1$
		[$n-2$
			[$n-3$,for descendants={edge=dotted}
				[]
				[]
			]
			[$\frac{n-2}{2}$,for descendants={edge=dotted}
				[]
				[]
			]
		]
		[$\frac{n-1}{2}$
			[$\frac{n-1}{2}-1$,for descendants={edge=dotted}
				[]
				[]
			]
			[$\frac{n-1}{4}$,for descendants={edge=dotted}
				[]
				[]
			]
		]
	][$\frac{n}{2}$
		[$\frac{n}{2}-1$
			[$\frac{n}{2}-2$,for descendants={edge=dotted}
				[]
				[]
			]
			[$\frac{\frac{n}{2}-1}{2}$,for descendants={edge=dotted}
				[]
				[]
			]
		]
		[$\frac{n}{4}$
			[$\frac{n}{4}-1$,for descendants={edge=dotted}
				[]
				[]
			]
			[$\frac{n}{8}$,for descendants={edge=dotted}
				[]
				[]
			]
		]
	]
]
% Left side arrow thingy.
\node(leftbox) at ($(current bounding box.west) - (1, 0)$){$n$};
\path (current bounding box.north west) -- (current bounding box.north) node(lefttopline){};
\draw[->] (leftbox) -- (leftbox |- lefttopline);
\path (current bounding box.south west) -- (current bounding box.south) node(leftbotline){};
\draw[->] (leftbox) -- (leftbox |- leftbotline);
% Right side arrow thingy.
\node(rightbox) at ($(current bounding box.east) + (1, 0)$){$\log_2(n)$};
\path (current bounding box.north east) -- (current bounding box.north) node(righttopline){};
\draw[->] (rightbox) -- (rightbox |- righttopline);
\path (current bounding box.south east) -- (current bounding box.south) node(rightbotline){};
\draw[->] (rightbox) -- (rightbox |- rightbotline);
\end{forest}
\end{center}

The left side of the tree has height $n$ while the right side of the tree has height $\log_2(n)$.  It's not clear to me what the height of the inner nodes is or how many leaf nodes there are.

I made a number of guesses regarding a good asymptotic upper bound, but those ended up ruling functions out rather than revealing a tight upper bound.

For example, consider any polynomial $O(n^x)$ where $x$ is a constant greater than 1.  Assume $T(n) \leq cn^x$ for all $m < n$, in particular $m = n - 1$ and $m = n/2$, then $T(n-1) \leq c(n-1)^x$ and $T(n/2) \leq c(n/2)^x$.  Substituting into the recurrence gives:
\begin{eqnarray*}
	T(n) & \leq & c(n-1)^x + c(n/2)^x + n \\
	 & = & \frac{(2^x + 1)cn^x}{2^x} + \ldots \text{lower-order terms}
\end{eqnarray*}
No matter what valid $x$ is chosen, the highest order term $\frac{c(2^x + 1)n^x}{2^x} > cn^x$ because it's always at least $\frac{cn^x}{2^x}$ larger than $cn^x$.  This led me to expect that the upper bound was exponential rather than polynomial...

For the next example, consider any exponential $O(x^n)$ where $x$ is a constant greater than 1.  Assume $T(n) \leq cx^n$ for all $m < n$, in particular $m = n - 1$ and $m = n/2$, then $T(n-1) \leq cx^{n-1}$ and $T(n/2) \leq cx^{n/2}$.  Substituting into the recurrence gives:
\begin{eqnarray*}
	T(n) & \leq & cx^{n-1} + cx^{n/2} + n
\end{eqnarray*}
No matter what valid $x$ is chosen, the recurrence will always be less than $cx^n$ as $n \rightarrow \infty$.

Searching online to find some function between polynomial and exponential gave $O(n^{\log_2 n})$ as a possibility.  Assume $T(n) \leq cn^{\log_2 n}$ for all $m < n$, in particular $m = n - 1$ and $m = n/2$, then $T(n-1) \leq c(n-1)^{\log_2(n - 1)}$ and $T(n/2) \leq c(n/2)^{\log_2 (n/2)}$.  Substituting into the recurrence gives:
\begin{eqnarray*}
	T(n) & \leq & c(n-1)^{\log_2(n-1)} + c(n/2)^{\log_2 (n/2)} + n \\
	 & = & c(n-1)^{\log_2(n-1)} + c(n/2)^{\log_2 n - 1} + n \\
	 & = & c(n-1)^{\log_2(n-1)} + c(n/2)^{\log_2 n}(n/2)^{-1} + n \\
	 & = & c(n-1)^{\log_2(n-1)} + c\frac{n^{\log_2 n}}{n}\frac{2}{n} + n \\
	 & = & c(n-1)^{\log_2(n-1)} + cn^{\log_2 n}(2/n^2) + n \\
	 & = & c(n-1)^{\log_2(n-1)} + c2n^{\log_2 n - 2} + n
\end{eqnarray*}

I believe that the sum of three above terms is less than $cn^{\log_2 n}$.

There are also possibilities such as $O(n^{\log_2 \log_2 n})$, but I've spent enough time on this problem for now.

My guess is that the recurrence is $O(n^{\log_2 n})$.

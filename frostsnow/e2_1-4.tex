It is not clear what the exercise means by the "$n$-element arrays $A$ and $B$"; are these supposed to be $n$-element arrays of $n$-bit integers or are they supposed to be $n$-element arrays of the bits of two respective $n$-bit integers? The latter is assumed.
\\ \\
\noindent~\textbf{Input}: Two sequences of $n$-bit arrays $A$ and $B$, each representing an $n$-bit integer.
\\ \\
\noindent~\textbf{Output}: An $(n + 1)$-bit array $C$ representing the addition of $A$ and $B$.
\\ \\
\noindent~\textbf{Pseudocode}:
\begin{verbatim}
carry = 0;
i = 0;
for (i = 0; i < A.length; i++) {
    temp = A[i] + B[i] + carry;
    carry = temp >> 1;
    C[i] = temp & 0x01;
}
C[i] = carry >> 1;
\end{verbatim}

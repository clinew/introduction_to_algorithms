\documentclass{article}

\usepackage{amsmath}

\begin{document}

\section*{Exercise 6.1-7}

A node is a leaf node if it has no children. If a node has a right child, then it has a left child, so it is only necessary to check for the absence of a left child to determine that a node is a leaf node; recall that the location of a left child for a given node $i$ is given by the function $\text{LEFT}(i) = 2i$. For an array of $n$-elements a node is a leaf node if both $i \leq n$ (the node $i$ is within the bounds of the array) and (the array has no left child):
\begin{eqnarray*}
	\text{LEFT}(i) & > & n \\
	2i & > & n \\
	i & > & \lfloor n / 2 \rfloor
\end{eqnarray*}
\ldots which gives the series $i = \lfloor n / 2 \rfloor$ + 1, $\lfloor n / 2 \rfloor + 2$, \ldots, $n$.

\end{document}

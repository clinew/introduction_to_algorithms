In order to prove that $T(n) = T(n - 1) + n$ is $O(n^2)$, attempt to use the Substitution Method on the bound as-is by trying to prove $T(n) \leq cn^2$ for an appropriate constant $c > 0$.
Assume that this bound holds for all $m < n$, in particular for $m = n - 1$, in which case:
\begin{eqnarray*}
	T(n - 1) &\leq& c(n - 1)^2
\end{eqnarray*}
Substituting into the recurrence then yields:
\begin{eqnarray*}
	T(n) &\leq& c(n - 1)^2 + n \\
	 &\leq& c(n^2 - 2n + 1) + n
\end{eqnarray*}
...which is basically impenetrable, so another guess is warranted. This time try to prove that $T(n) \leq cn^2 + cn$ for an approrpiate constant $c > 0$, again assuming that this bound holds for all $m < n$ and, in particular, for $m = n - 1$, gives:
\begin{eqnarray*}
	T(n - 1) &\leq& c(n - 1)^2 + c(n - 1)
\end{eqnarray*}
Substituting into the recurrence then yields:
\begin{eqnarray*}
	T(n) &\leq& c(n - 1)^2 + c(n - 1) + n \\
	 &\leq& cn^2 - 2cn + c + cn - c + n \\
	 &\leq& cn^2 - cn + n \\
	 &\leq& cn^2 + (-c - 1)n \\
	 &\leq& cn^2 + cn
\end{eqnarray*}
where the last step holds as long as $c \geq \frac{1}{2}$.
For the Base Case, use the boundary condition by assuming that $T(1) = 1$. Then we have:
\begin{eqnarray*}
	T(1) &\leq& cn^2 + cn \\
	 &\leq& c1^2 + c1 \\
	 &\leq& 2c \\
	1 &\leq& 2c
\end{eqnarray*}
...which holds for all $c \geq \frac{1}{2}$.
Thus, $T(n) = T(n - 1) + n$ is $O(n^2)$.

% Base Case of T(8) == 8 rather than T(5) = 1

No base case for the recurrence was provided, but the recurrence will enter an infinite loop if it tries to converge on $T(4)$, so a base case of $T(8) = 8$ has been assumed.  This gives the recurrence tree:
% unannotated recurrence tree
\newlength{\leafnodelength}
\settowidth{\leafnodelength}{$\scriptscriptstyle n/4 +$}
\newsavebox{\leafnode}
\savebox{\leafnode}{\parbox{\leafnodelength}{$\scriptscriptstyle n/4 +$\vspace{-2mm} \\ $\scriptscriptstyle 1 + 2$}}
\begin{center}
\begin{forest}
[$n$, for tree={s sep=-1mm},name=level1
	[$n/2 + 2$
		[\usebox{\leafnode},for descendants={edge=dotted}
			[]
			[]
			[]
			[]
		][\usebox{\leafnode},for descendants={edge=dotted}
			[]
			[]
			[]
			[]
		][\usebox{\leafnode},for descendants={edge=dotted}
			[]
			[]
			[]
			[]
		][\usebox{\leafnode},for descendants={edge=dotted}
			[]
			[]
			[]
			[]
		]
	][$n/2 + 2$
		[\usebox{\leafnode},for descendants={edge=dotted}
			[]
			[]
			[]
			[]
		][\usebox{\leafnode},for descendants={edge=dotted}
			[]
			[]
			[]
			[]
		][\usebox{\leafnode},for descendants={edge=dotted}
			[]
			[]
			[]
			[]
		][\usebox{\leafnode},for descendants={edge=dotted}
			[]
			[]
			[]
			[]
		]
	][$n/2 + 2$
		[\usebox{\leafnode},for descendants={edge=dotted}
			[]
			[]
			[]
			[]
		][\usebox{\leafnode},for descendants={edge=dotted}
			[]
			[]
			[]
			[]
		][\usebox{\leafnode},for descendants={edge=dotted}
			[]
			[]
			[]
			[]
		][\usebox{\leafnode},for descendants={edge=dotted}
			[]
			[]
			[]
			[]
		]
	][$n/2 + 2$,name=level2
		[\usebox{\leafnode},for descendants={edge=dotted}
			[]
			[]
			[]
			[]
		][\usebox{\leafnode},for descendants={edge=dotted}
			[]
			[]
			[]
			[]
		][\usebox{\leafnode},for descendants={edge=dotted}
			[]
			[]
			[]
			[]
		][\usebox{\leafnode},for descendants={edge=dotted},name=level3
			[]
			[]
			[]
			[,name=level4]
		]
	]
]
\end{forest}
\end{center}
In order to determine the value of a node at a given height, assume an arbitrarily large $n$ and then apply the recurrence to it:
\begin{eqnarray*}
	\frac{\frac{n}{2} + 2}{2} + 2 & = & \frac{n}{4} + 1 + 2 = \frac{n}{4} + 3 \\
	\frac{\frac{n}{4} + 3}{2} + 2 & = & \frac{n}{8} + 1.5 + 2 = \frac{n}{8} + 3.5 \\
	\frac{\frac{n}{8} + 3.5}{2} + 2 & = & \frac{n}{16} + 1.75 + 2 = \frac{n}{16} + 3.75
\end{eqnarray*}
From this can be derived the formula for the value of a node at a given height $i$:
\begin{eqnarray*}
	\frac{n}{2^i} + 4 \sum^i_{x=1} \frac{1}{2^i}
\end{eqnarray*}
% level i total
Since a level at height $i$ will have $4^i$ total nodes, the total values of all the nodes at height $i$ is:
\begin{eqnarray*}
	4^i \left ( \frac{n}{2^i} + 4 \sum^i_{x=1} \frac{1}{2^i} \right ) & = & \frac{4^i n}{2^i} + 4^i 4 \sum^i_{x=1} \frac{1}{2^i} \\
	& = & \frac{4^i n}{2^i} + 4^i \cdot 4 \left ( -2 \left ( \frac{1}{2} \right )^{i+1} + 1 \right ) \\
	& = & 2^i n + 4^i \left ( -8 \left ( \frac{1}{2} \right )^{i+1} + 4 \right ) \\
	& = & 2^i n - 8 \cdot 4^i \left ( \frac{1}{2} \right )^{i+1} + 4 \cdot 4^i \\
	& = & 2^i n - 4 \cdot 2^i + 4 \cdot 4^i  \\
	& = & 2^i n + 4 \left (4^i - 2^i \right )
\end{eqnarray*}
% total tree height
Playing around with tree height given the base case of $n=8$ for the recurrence gives:
\begin{center}
\begin{tabular}{c|c}
	$n$ & $i$ \\
	8 & 0 \\
	12 & 1 \\
	20 & 2 \\
	36 & 3
\end{tabular}
\end{center}
From which can be derived the total height of the tree: $\log_2(n-4) - 2$.  The total for the leaves of the tree can then be calculated:
\begin{eqnarray*}
	n2^{\log_2(n-4)-2} + 4 \left ( 4^{\log_2(n-4)-2} - 2^{\log_2(n-4)-2} \right ) \\
	\frac{n 2^{\log_2(n-4)}}{4} + 4 \left ( \frac{4^{\log_2(n-4)}}{16} - \frac{2^{\log_2(n-4)}}{4} \right ) \\
	\frac{n(n-4)}{4} + 4 \left ( \frac{(n-4)^2}{16} - \frac{(n-4)}{4} \right ) \\
	\frac{n^2 -4n}{4} + 4 \left ( \frac{n^2 - 8n + 16}{16} - \frac{(n-4)}{4} \right ) \\
	\frac{n^2}{4} - n + \frac{n^2}{4} - 2n + 4 - n + 4 \\
	\frac{n^2}{2} - 4n + 8
\end{eqnarray*}
Using this data to properly annotate the recurrence tree gives:
\begin{center}
\settowidth{\leafnodelength}{$\scriptscriptstyle n/4 +$}
\savebox{\leafnode}{\parbox{\leafnodelength}{$\scriptscriptstyle n/4 +$\vspace{-2mm} \\ $\scriptscriptstyle 1 + 2$}}
\hspace*{-0.75in} % Push to the left a bit.
\begin{forest}
[$n$, for tree={s sep=-1mm},name=level1
	[$n/2 + 2$
		[\usebox{\leafnode},for descendants={edge=dotted}
			[]
			[]
			[]
			[]
		][\usebox{\leafnode},for descendants={edge=dotted}
			[]
			[]
			[]
			[]
		][\usebox{\leafnode},for descendants={edge=dotted}
			[]
			[]
			[]
			[]
		][\usebox{\leafnode},for descendants={edge=dotted}
			[]
			[]
			[]
			[]
		]
	][$n/2 + 2$
		[\usebox{\leafnode},for descendants={edge=dotted}
			[]
			[]
			[]
			[]
		][\usebox{\leafnode},for descendants={edge=dotted}
			[]
			[]
			[]
			[]
		][\usebox{\leafnode},for descendants={edge=dotted}
			[]
			[]
			[]
			[]
		][\usebox{\leafnode},for descendants={edge=dotted}
			[]
			[]
			[]
			[]
		]
	][$n/2 + 2$
		[\usebox{\leafnode},for descendants={edge=dotted}
			[]
			[]
			[]
			[]
		][\usebox{\leafnode},for descendants={edge=dotted}
			[]
			[]
			[]
			[]
		][\usebox{\leafnode},for descendants={edge=dotted}
			[]
			[]
			[]
			[]
		][\usebox{\leafnode},for descendants={edge=dotted}
			[]
			[]
			[]
			[]
		]
	][$n/2 + 2$,name=level2
		[\usebox{\leafnode},for descendants={edge=dotted}
			[]
			[]
			[]
			[]
		][\usebox{\leafnode},for descendants={edge=dotted}
			[]
			[]
			[]
			[]
		][\usebox{\leafnode},for descendants={edge=dotted}
			[]
			[]
			[]
			[]
		][\usebox{\leafnode},for descendants={edge=dotted},name=level3
			[]
			[]
			[]
			[,name=level4]
		]
	]
]
% Bottom leaves.
\node(bottombox) at ($(current bounding box.south) - (0, 1)$){};
\foreach \x in {-5, -4, ..., 2, 4, 5} {
	\node(leaf\x) at ($(bottombox.west) + (\x, 0)$){$T(8)$};
	\draw[dotted] (leaf\x) -- +(0, 1);
}
\node at ($(bottombox.west) + (3, 0)$){\ldots};
% Left side arrow thingy.
\node(leftbox) at ($(current bounding box.west) - (1.25, 0)$){$\log_2(n - 4) - 2$};
\path (current bounding box.north west) -- (current bounding box.north) node(topline){};
\draw[->] (leftbox) -- (leftbox |- topline);
\path (current bounding box.south west) -- (current bounding box.south) node(botline){};
\draw[->] (leftbox) -- (leftbox |- botline);
% Right side sums.
\path ($(current bounding box.north east) + (0.5, 0)$) -- ($(current bounding box.south east) + (0.5, 0)$) node(rightline){};
\draw[-stealth,dotted,thick] (level1) -- (level1 -| rightline) node(l1node){};
\node at ($(l1node) + (1, 0)$){$n$};
\draw[-stealth,dotted,thick] (level2) -- (level2 -| rightline) node(l2node){};
\node at ($(l2node) + (1, 0)$){$2n + 8$};
\draw[-stealth,dotted,thick] (level3) -- (level3 -| rightline) node(l3node){};
\node at ($(l3node) + (1, 0)$){$4n + 48$};
\path[] (level4) -- (level4 -| rightline) node(l4node){};
\node at ($(l4node) + (1, 0)$){\vdots};
\draw[-stealth,dotted,thick] (leaf5) -- (leaf5 -| rightline) node(leafnode){};
\node (rightleaftotal) at ($(leafnode) + (1, 0)$){$\frac{n^2}{2} - 4n + 8$};
\end{forest}
\end{center}
Summing up the total for all levels gives:
\begin{eqnarray*}
	\hspace*{-1in} % Push to the left a bit.
	\sum_{i=0}^{\log_2(n-4)-2} \left ( 2^i n + 4 \left ( 4^i - 2^i \right ) \right ) & = & n \sum_{i=0}^{\log_2(n-4)-2} 2^i + 4 \sum_{i=0}^{\log_2(n-4)-2} 4^i - 4 \sum_{i=0}^{\log_2(n-4)-2} 2^i \\
	& = & n \left ( \frac{2^{\log_2(n-4)-1} - 1}{1} \right ) + 4 \left ( \frac{4^{\log_2(n-4)-1} - 1}{3} \right ) - 4 \left ( \frac{2^{\log_2(n-4)-1} - 1}{1} \right ) \\
	& = & n 2^{\log_2(n-4)-1} - n + \frac{4 \cdot 4^{\log_2(n-4)-1}}{3} - \frac{4}{3} - 4 \cdot 2^{\log_2(n-4)-1} + 4 \\
	& = & \frac{n 2^{\log_2(n-4)}}{2} - n + \frac{4^{\log_2(n-4)}}{3} - \frac{4}{3} - 2 \cdot 2^{\log_2(n-4)} + 4 \\
	& = & \frac{n(n-4)}{2} - n + \frac{(n-4)^2}{3} - \frac{4}{3} - 2(n-4) + 4 \\
	& = & \frac{n^2}{2} - 2n - n + \frac{n^2}{3} - \frac{8n}{3} + \frac{16}{3} - \frac{4}{3} - 2n + 8 + 4 \\
	& = & \frac{5n^2}{6} - \frac{23n}{3} + 16
\end{eqnarray*}
% Prove inductive case
This gives a guess of $O(n^2)$.  In order to verify, use the Substitution Method to show that $T(n) \leq cn^2 - 8cn -n$; assume that this is true for all $m < n$, in particular for $m = \frac{n}{2} + 2$, then $T(\frac{n}{2} + 2) \leq c(\frac{n}{2} + 2)^2 - 8c(\frac{n}{2} + 2) - (\frac{n}{2} + 2)$; substituting this into the recurrence equation then gives:
\begin{eqnarray*}
	T(n) & \leq & 4 \left ( c \left ( \frac{n}{2} + 2 \right )^2 - 8c \left ( \frac{n}{2} + 2 \right ) - \left ( \frac{n}{2} + 2 \right ) \right ) + n \\
	& = & 4 \left ( \frac{cn^2}{4} + 2cn + 4c - 4cn - 16c - \frac{n}{2} - 2 \right ) + n \\
	& = & 4 \left ( \frac{cn^2}{4} - 2cn - 12c - \frac{n}{2} - 2 \right ) + n \\
	& = & cn^2 - 8cn - n - 48c - 8 \\
	& \leq & cn^2 - 8cn - n
\end{eqnarray*}
Thus the Inductive Case is proven.  For the Base Case, note that the recurrence doesn't have a clear definition when $n$ has a fractional part.  One can use the above total for all levels for $8 < n < 12$ to compensate for this, but this proof will ignore those subtleties and focus on whole values.  Trying to prove the Base Case for $T(8) = 8$ gives:
\begin{eqnarray*}
	T(8) & \leq & cn^2 - 8cn - n \\
	8 & \leq & c8^2 - 8c(8) - 8 \\
	8 & \leq & 64c - 64c - 8 \\
	8 & \leq & -8
\end{eqnarray*}
\ldots which is clearly false for any value of $c$.  Using the larger values of $n = 12$ and $n = 20$ for the base cases of the inductive proof gives:
\begin{eqnarray*}
	T(12) & \leq & c(12)^2 - 8c(12) - 12 \\
	44 & \leq & 48c - 12 \\
	T(20) & \leq & c(20)^2 - 8c(20) - 20 \\
	196 & \leq & 240c - 20
\end{eqnarray*}
Where the base case for $n = 12$ holds for $c \geq 1 \frac{1}{6}$ and the base case for $n = 20$ holds for $c \geq \frac{9}{10}$.  Thus for any values $c \geq 1 \frac{1}{6}$ the base cases hold.

Therefore the recurrence is $O(n^2)$.
